\documentclass{article}
\usepackage{graphicx} % Required for inserting images

\title{Polynomial Multiplication with Parallelized FFT writeup}
\author{Alex Pan}
\date{April 2024}

\usepackage[margin=1in]{geometry}
\usepackage{amsmath}
\usepackage{amsfonts}
\usepackage{hyperref}
\usepackage{parskip}

\begin{document}

\maketitle

\section{Overview}
Using the Fast Fourier Transform (FFT), we can multiply two univariate polynomials of in $O(n\log(n))$ time, compared to the trivial distributive algorithm of $O(n^2)$. In addition, this process is fully parallelizable, making its computation extremely quick using GPUs. In order to understand the algorithm, we need to understand a few steps:

\begin{itemize}
    \item Polynomial Interpolation 
    \item DFT and FFT
    \item Parallelization
\end{itemize}

\section{Polynomial Interpolation}
Given a set of $n+1$ points $(x_0,y_0), (x_1,y_1) ... (x_n, y_n)$, with all $x_i$ being distinct, there exists a unique polynomial of degree \textit{at most} $n$ that passes through all $n+1$ points.

To verify this, we start by writing the polynomial $p(x)$ as $a_0 + a_1x^1 + a_2x^2 + ... + a_nx^n$, we can write this as a matrix product:

\centerline{$\begin{bmatrix} 1 & x & x^2 & \ldots & x^n \end{bmatrix}\begin{bmatrix} a_0 \\ a_1 \\ a_2 \\ \vdots \\ a_n \end{bmatrix} = p(x) = y$}

Doing this for all $x_i$, we get the \textit{Vandermonde Matrix} and the equation:

\centerline{
    $
    \overbrace{
        \begin{bmatrix}
            1 & x_0 & x_0^2 & \ldots & x_0^n \\
            1 & x_1 & x_1^2 & \ldots & x_1^n \\
            1 & x_2 & x_2^2 & \ldots & x_2^n \\
            \vdots & \vdots & \vdots & \ddots & \vdots \\
            1 & x_n & x_n^2 & \ldots & x_n^n \\
        \end{bmatrix}
    }^{\text{\large $V$}}
    \overbrace{
        \begin{bmatrix}
            a_0 \\ a_1 \\ a_2 \\ \vdots \\ a_n 
        \end{bmatrix}
    }^{\text{\large $\vec{a}$}} = 
    \begin{bmatrix}
        p(x_0) \\ p(x_1) \\ p(x_2) \\ \vdots \\ p(x_n) 
    \end{bmatrix} = 
    \overbrace{
        \begin{bmatrix}
            y_0 \\ y_1 \\ y_2 \\ \vdots \\ y_n 
        \end{bmatrix}
    }^{\text{\large $\vec{y}$}}
    $
}

To show that $\vec{a}$ is exists and is unique, we need to show that the Vandermonde matrix is bijective. Because it is square, we just need to verify that the matrix is invertible. From \href{https://en.wikipedia.org/wiki/Vandermonde_matrix}{Wikipedia}, the determinant of this matrix is nonzero if and only if all $x_i$ are distinct, which we specified earlier. So, the polynomial represented by $\vec{a}$ exists and is unique, with $\vec{a}=V^{-1}\vec{y}$. (The non-linear algebra proofs are a lot more interesting)

An important comment to make is that the degree of the interpolated polynomial doesn't have to be $n$; the degree d can be any value from $1\le d\le n$ this is the case when $a_{d+1} \ldots a_n=0$. This means that given a set of data points, the interpolated polynomial will be of the lowest degree possible.

If we evaluate two polynomials $p(x)$ and $q(x)$ at a point $x_0$, then multiply their outputs together, we get the same value as if we multiplied the two polynomials first, then evaluated them at $x_0$. This forms the motivation for using polynomial interpolation to multiply polynomials.

The algorithm for multiplying polynomials $p(x)$ and $q(x)$, with coefficients $\vec{a}$ and $\vec{b}$ using interpolation is as follows:

\begin{enumerate}
    \item Determine degree $k$ of product: This is just the degrees of $p_1(x)$ and $p_2(x)$ added together: $n+m=k$
    \item Select arbitrary $k+1$ distinct values $x_0, x_1, \ldots, x_k$. Evaluate the polynomials at all of the values using the $(k+1)\times (k+1)$ Vandermonde matrix: $V\vec{a}=\vec{y}$, $V\vec{b}=\vec{z}$
    \item Perform component-wise multiplication on $\vec{y}$ and $\vec{z}$:

    \centerline{
        $
        \begin{bmatrix}
            y_0 \\ y_1 \\ y_2 \\ \vdots \\ y_k 
        \end{bmatrix} \circ
        \begin{bmatrix}
            z_0 \\ z_1 \\ z_2 \\ \vdots \\ z_k 
        \end{bmatrix} = 
        \begin{bmatrix}
            y_0z_0 \\ y_1z_1 \\ y_2z_2 \\ \vdots \\ y_kz_k 
        \end{bmatrix}
        $
    }
    \item Interpolate $\vec{y}\circ\vec{z}$ - find the unique polynomial that passes through all of the chosen points - compute $V^{-1}(\vec{y}\circ\vec{z})$
\end{enumerate}
This algorithm works, but runs in $O(n^3)$ time because finding the inverse of a matrix is $O(n^3)$, making it significantly slower than trivial multiplication. DFT fixes the finding the inverse part.

\section{DFT and FFT}
\subsection{DFT}
The Discrete Fourier Transform is a function maps a vector containing elements $[a_0, a_1, ... a_n]$ with $a_i\in \mathbb{C}$ to a vector containing elements $[\hat{a}_0, \hat{a}_1, ... \hat{a}_n]$ with the following rule:

$\hat{a}_k = \displaystyle\sum\limits_{j=0}^{n-1} x_j\cdot e^{-2\pi i \frac{k}{N}j}$

In the context of polynomial multiplication, DFT just evaluates a polynomial of degree $n$ at the $n$ complex roots of unity. So, with $\omega = e^{\frac{2\pi i}{n}}$ the Vandermonde matrix looks like this:

\centerline{
    $
    \overbrace{
        \begin{bmatrix}
            1 & 1 & 1 & \ldots & 1 \\
            1 & \omega & \omega_0^2 & \ldots & \omega_0^{n-1} \\
            1 & \omega^2 & \omega_0^4 & \ldots & \omega_0^{2(n-1)} \\
            \vdots & \vdots & \vdots & \ddots & \vdots \\
            1 & \omega^{n-1} & \omega_0^{2(n-1)} & \ldots & \omega_0^{(n-1)(n-1)} \\
        \end{bmatrix}
    }^{\text{\large $V$}}
    \overbrace{
        \begin{bmatrix}
            a_0 \\ a_1 \\ a_2 \\ \vdots \\ a_n 
        \end{bmatrix}
    }^{\text{\large $\vec{a}$}} = 
    \overbrace{
        \begin{bmatrix}
            y_0 \\ y_1 \\ y_2 \\ \vdots \\ y_n 
        \end{bmatrix}
    }^{\text{\large $\vec{y}$}}
    $
}

DFT is doing the same thing as the first step in the above algorithm, except it specifies the arbitrary points to be the $k$ roots of unity.

This is only useful because the inverse Vandermonde matrix $V^{-1}$ is already known - we don't need to calculate it:

\centerline{
    $ V^{-1} = \displaystyle\frac{1}{n}
    \begin{bmatrix}
        1 & 1 & 1 & \ldots & 1 \\
        1 & \omega^{-1} & \omega_0^{-2} & \ldots & \omega_0^{-(n-1)} \\
        1 & \omega^{-2} & \omega_0^{-4} & \ldots & \omega_0^{-2(n-1)} \\
        \vdots & \vdots & \vdots & \ddots & \vdots \\
        1 & \omega^{-1(n-1)} & \omega_0^{-2(n-1)} & \ldots & \omega_0^{-(n-1)(n-1)} \\
    \end{bmatrix}
    $
}

To verify: Checking entries on the diagonal, multiplying $\omega$ by its inverse $\omega^{-1}$ gives 1, and adding these up $n$ times and dividing by $n$ gives 1's on the diagonal. Checking entries not on the diagonal and in the first row or column, we see that adding the roots of unity just gives 0. I don't feel like proving the other entries.

With this, we no longer have to compute the inverse of the Vandermonde matrix in $O(n^3)$ time, and our algorithm has been shortened to $O(n^2)$, the complexity of multiplying a matrix by a vector. However, since we're doing so many extra steps, this algorithm is still slower, and requires more memory than the trivial algorithm.

\subsection{FFT}

\end{document}
