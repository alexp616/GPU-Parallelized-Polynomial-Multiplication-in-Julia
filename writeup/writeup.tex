\documentclass{article}
\usepackage{graphicx} % Required for inserting images

\title{Raising Polynomials to Powers}
\author{Alex Pan}
\date{April 2024}

\usepackage[margin=1in]{geometry}
\usepackage{amsmath}
\usepackage{amsfonts}
\usepackage{hyperref}
\usepackage{parskip}
\usepackage{wasysym}
\usepackage{adjustbox}

\begin{document}

\maketitle

Proofs for correctness of computed numbers:

Because the GPU doesn't report when an integer datatype overflows, we need to verify that the numbers the algorithm returns are actually correct, and less than $2^64$.

\textit{Theorem: } Let $f$ be a $k$-variate polynomial with homogeneous degree $n$. Let $M$ be the coefficient of $f^p$. Then, the maximum coefficient of $f^p$ satisfies $M' \le M^p \cdot \binom{n + k - 1}{k - 1}^{p-1}$

\textit{Proof: } Say we have a $k$-variate polynomial $f$ with homogeneous degree $n$. Each term of $f$ will look like $cx_{1}^{d_1}x_{2}^{d_2}...x_{k}^{d_k}$, with $ \sum_{1}^{k} d_i = n$. To make things easier, let $(d_1, d_2, ... , d_k)$ denote the degree of a term from now on.

Let $(d'_1, d'_2, ... , d'_k)$ denote the degree of a term of $f^2$. To find all unreduced terms that contribute to it, we look at pairs of terms of $f$ with the original degrees $(d'_1 - a_1, d'_2 - a_2, d'_3 - a_3, ... , d'_k - a_k)$ and $(a_1, a_2, a_3, ... , a_k)$, where $a_i > 0$ and $\sum_{1}^{k} = n$. The number of these terms is bounded above by the number of weak integer compositions of $n$ into $k$ parts, which is given by $\binom{n + k - 1}{k - 1}$. The coefficients of each of these resulting terms is bounded above by $M^2$, and because the number of these is bounded above by $\binom{n + k - 1}{k - 1}$, we know that the maximum reduced coefficient of $f^2$ is $M^2 \cdot \binom{n + k - 1}{k - 1}$

The inductive step is similar. Let $(d'_1, d'_2, ... , d'_k)$ denote the degree of a term of $f^p$. The pairs of terms that contribute to it come respectively from $f^{p - 1}$ and $f$, so we look again at a term of $f^{p - 1}$ with degree $(d'_1 - a_1, d'_2 - a_2, d'_3 - a_3, ... , d'_k - a_k)$, and at a term of $f$ with degree $(a_1, a_2, a_3, ... , a_k)$. Again, there are at most $\binom{n + k - 1}{k - 1}$ of these, and because the maximum unreduced coefficient is bounded above by $\left(M^{p - 1} \cdot \binom{n + k - 1}{k - 1} ^ {p - 2}\right) \cdot (M)$, we have an upper bound of $M^p \cdot \binom{n + k - 1}{k - 1}^{p-1}$ for the coefficients. QED

We are interested in two steps: raising $g = f^{p - 1}\mod p$ for primes $p$, and raising $g ^ p$ for the same $p$.

\begin{itemize}
    \item $g = f^{p - 1}\mod p$
    
    If $f$ is a polynomial of 4 variables, then we have an upper bound of $(p - 1)^{p-1} \cdot \binom{4 + 4 - 1}{4 - 1}^{p - 1}$. This is less than $2^{64}$ for primes $5$ and $7$, so we can reduce mod $p$ at the end of our computation without overflowing. (In the raise to $p-1$ case, the \href{https://oeis.org/A333901/internal}{OEIS} sequence gives a better bound, but I don't know how to prove that the problems are equivalent yet. That bound is actually obtainable)

    If $f$ is a polynomial of 5 variables, then primes $5$ and $7$ still work for this first step.

    \item $g ^ p$
    
    If $f$ is a polynomial of 4 variables, and it has been raised to the $p-1$ and taken mod $p$, then the greatest coefficient of $g$ is $p-1$, and $g$ is homogeneous with degree $4 * p$. This gives the upper bound $(p - 1)^{p} \cdot \binom{4p + 4 - 1}{4 - 1}^{p}$. For $p=5$, this evaluates to $1.00733564 \cdot 10^{17}$, which is less than $2^{64} - 1$, or $1.84467441 \cdot 10^{19}$. So, for the case of 4 variables, and $p = 5$, Int64's can be used without having to worry about overflowing.
\end{itemize}

\section*{Using FFT to raise polynomials to powers}
DFT works for multiplying polynomials by evaluating both polynomials at $n$ points, represented by the output vectors $\hat{f}$ and $\hat{g}$. Component-wise multiplication is then performed on $\hat{f}$ and $\hat{g}$, representing the product polynomial evaluated at $n$ points, and IDFT is performed to give the actual coefficients of the polynomial.

If we want to raise a polynomial to a power, we can simply DFT to get $\hat{f}$, then raise each element of $\hat{f}$ to the $p$-th power, and that will represent $f^p$ evaluated at $n$ points, and IDFT to get the coefficients of $f^p$.

This should be much faster than the repeated squaring method; this method only requires one DFT to be evaluated, and raising each component to a power in parallel should be effectively instant. In contrast to $log_2(p)$ different DFTs
\end{document}

